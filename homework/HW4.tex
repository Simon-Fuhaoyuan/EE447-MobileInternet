\documentclass{article}
\usepackage[UTF8]{ctex}
\usepackage{CJKutf8}
\usepackage{color}
\usepackage{natbib}
\usepackage{graphicx}
\usepackage{subfigure}
\usepackage{geometry}
\usepackage{amsmath}
\usepackage{mathrsfs}
\usepackage{algorithm}  
\usepackage{algorithmicx}  
\usepackage{algpseudocode}
\usepackage[backref]{hyperref}
\usepackage{diagbox}
\usepackage{pythonhighlight}
\hypersetup{hidelinks}
\renewcommand{\algorithmicrequire}{\textbf{Input:}}  % Use Input in the format of Algorithm  
\renewcommand{\algorithmicensure}{\textbf{Output:}}
\geometry{a4paper, scale=0.8}

\title{\textbf{EE447 Homework4 \\ Multi-hop Neighbours in ER Networks}}
\author{付昊源 517021910753}
\date{May 22, 2020}

\begin{document}

\maketitle

%%%%%%%%%%%%%%%%%%%%%%%%%%%%%%
\section*{Problem Statement}
In an ER network, we can consider a graph $G=\mathcal{G} (n, p)$ as a random graph, i.e. each edge in $G$ has the probability $p$ of being present and $(1-p)$ of being absent, independently of the other edges. Besides, we further sample the edges in $G$ for two times to construct two new graphs $G_1$ and $G_2$. In each of the new graph, each edge of $G$ has the probability $s$ of being preserved and naturally $(1-s)$ of being deleted, and $G_1$ and $G_2$ are sampled independently with each other. Thus, we can formulate two new graphs as $G_1=\mathcal{G}(n, ps)$, $G_2=\mathcal{G}(n, ps)$. Note that $G_1$ and $G_2$ are related with $G$, and there cannot be any new edge in $G_1$ or $G_2$. All these relationships can be formulated as
\begin{equation}
    \begin{aligned}
    E(G) \supseteq E(G_1) \cup E(G_2) \\
    V(G)=V(G_1)=V(G_2)
    \end{aligned}
\end{equation}
where $E()$ means the edge set and $V()$ means the vertices set.

We define that two different nodes $u$ and $v$ in network $G$ are 2-hop neighbors if and only if their shortest distance in $G$ is exactly 2. Provided that $0 < s \le 1$ and $p \ll 1 \ll np$, we are assigned to prove the following statements:
\begin{enumerate}
    \item The summation of expected number of 2-hop neighbors for all nodes in network $G$ can be approximated by $n^3p^2$.
    \item The probability of nodes $u$, $v$ being 2-hop neighbors in both sampled networks $G_1$ and $G_2$ if they are 2-hop neighbors in network $G$ can be approximated by $s^4$.
\end{enumerate}

\section*{Proof of statement 1}
Suppose the set of vertices is $\{v_1, v_2, \cdots, v_n\}$ and we use $P_{ij}$ to denote the probability of an edge from $v_i$ to $v_j$ exists. Clearly, $P_{ij}=P_{ji}=p$. We further use $V_i^1$ and $V_i^2$ to denote the set of 1-hop neighbors and the set of 2-hop neighbors for vertex $v_i$, respectively.

For an arbitrary vertex $v_i$ in $G$, the expected number of items in $V_i^1$ is
\begin{equation}
    E(|V_i^1|)=\sum_{j\neq i}1\cdot P_ij=(n-1)\cdot p \approx np
\end{equation}
since $np\gg p$.

We first calculate the number of items in $V_i^1$ for the following two properties. One is that, if a vertex $v_j$ is the 1-hop neighbor of $v_i$, then it cannot be 2-hop neighbor of $v_i$. The other is, if $v_j$ is the 2-hop neighbor of $v_i$, then there must exist $v_k$, who is 1-hop neighbor of both $v_i$ and $v_j$. These two properties both relate to 1-hop neighbors, so we can further calculate the expected number of 2-hop neighbors for a given vertex $v_i$ based on $E(|V_i^1|)$:
\begin{equation}
    E(|V_i^2|)=\sum_{j\neq i, j\notin V_i^1}\sum_{k\in V_i^1}1\cdot P_{jk}=(n-np-1)np\cdot p\approx n^2p^2
\end{equation}

Thus, for all vertices in network $G$, the summation of expected number of 2-hop neighbors can be approximated by $n\cdot n^2p^2=n^3p^2$

\section*{Proof of statement 2}
According to the setting in statement 2, node $u$ and $v$ have already been 2-hop neighbors in graph $G$, which is the given condition. We can construct a set $V_{uv}\subset V(G)$, which contains all nodes satisfying
\begin{equation}
    \forall v_i\in V_{uv},\quad \{u,v_i\}\in E(G) \land  \{v_i,v\}\in E(G)
\end{equation}

If node $u$ and $v$ are still 2-hop neighbors in sampled graph $G_1$ and $G_2$, the intermediate node connecting $u$ and $v$ must also exist in $V_{uv}$. Without loss of generality, assume
\begin{equation}
    \begin{aligned}
    v_{i_1}\in V_{uv},\quad \{u,v_{i_1}\}\in E(G_1) \land  \{v_{i_1},v\}\in E(G_1) \\
    v_{i_2}\in V_{uv},\quad \{u,v_{i_2}\}\in E(G_2) \land  \{v_{i_2},v\}\in E(G_2)
    \end{aligned}
\end{equation}
In this way, we only care about the appearance or absence of the four edges in equation 5. If $u$ and $v$ are 2-hop neighbors both in $G_1$ and $G_2$, the four edges above must all exist. Due to the independence of sampling, the probability is $s\cdot s\cdot s\cdot s=s^4$.
%%%%%%%%%%%%%%%%%%%%%%%%%%%%%%

\end{document}
